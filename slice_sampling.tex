\section{Slice sampling}

Slice sampling je metóda generovania z triedy Metropolis-Hastings algoritmov, ktorá je vhodná na generovanie pre distribúcie s určitou štruktúrou.
Majme $n$ rozmernú hustotu $\pi: \mathbb{R}^n \rightarrow [0,\infty)$, pričom $\pi(x) = \Pi^K_{j=0}f_j(x)$ pre nejaké $f_j: \mathbb{R}^n \rightarrow [0,\infty)$.

Slice sampler bude pracovať nasledovne. Začne s bodom $x^{(0)}$, na vygenerovanie bodu $x^{(i)}$ z $x^{(i-1)}$ najprv vygeneruje nezávislé náhodné premenné $y_{i,j}$ v závislosti od $f_j(x^{(i-1)})$. Následne pomocou $y_{i,j}$ vygeneruje bod $x^{(i)}$.

\begin{algorithm}[H]
	\caption{Slice sampling algoritmus \cite{slice_convergence_roberts}}
	\label{slice}
	\begin{algorithmic}[1]
		\State inicializuj $x^{(0)}$
		\For {$i=1,\dots,N$}
			\State vygeneruj nezávislé náhodné premenné $y_{i,1},y_{i,2},\dots, y_{i,K}$,

			kde $y_{i,j} \sim U(0,f_j(x^{(i-1)}))$
			\State vygeneruj $x^{(i)}$ z distribúcie $f_0(.)\mathbf{1}_{L(y_{i})}$,

			kde $L(y)= \{ z \in \mathbb{R}^n; f_j(z) \geq y_{i,j}, i=1,2, \dots, K\}$
		\EndFor
		\State Vráť ${x^{(1)},x^{(2)},\dots,x^{(N)}}$.
	\end{algorithmic}
\end{algorithm}

Daný algoritmus je závislý jedine od faktorizácie distribúcie $\pi(x)$ na $\Pi^K_{j=0}f_j(x)$. \textbf{TODO rozvinut}

Daný algoritmus je tiež špeciálnym prípadom Metropolis-Hastings algoritmu, možno ho analyzovať rovnakým spôsobom.\\

\textbf{TODO} porovnanie so vseobecnym MH a Gibbsom

\textbf{TODO} vyuzitie pri polyedroch
