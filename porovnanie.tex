\chapter{Porovnanie generátorov}

Táto kapitola bude obsahovať praktické porovnania algoritmov spomenutých v prvej kapitole a zdrojový kód čo najrýchlejšieho algoritmu na generovanie bodov vnútri polyédru.

Predpokladáme, že najrýchlejší generátor bude využívať MVEE elipsoid, teda táto kapitola bude obsahovať taktiež implementáciu REX algoritmu.\\

Pre účely tejto práce budeme pracovať s polyédrom reprezentovaným sústavou lineárnych nerovníc, riešení systému $Ax \leq b$ ($x \in X$ ak $Ax \leq b$).
Ako základ náhody bude náš generátor bodu v polyédre používať rovnomerný generátor čísel na $[0,1]$ (ďalej $U[0,1]$). Pomocou $U[0,1]$ možno triviálne generovať bod na $[0,k]$ (prenásobením konštantou $k$), tiež možno generovať bod na $[a,b]$ (vygenerovaním bodu na $[0, -a+b]$ a pripočítaním konštanty $a$, alebo bod na $[0,1]^n$ (postupným vygenerovaním súradníc). Generovanie na iných polyédroch, najmä v priestoroch vysokej dimenzie, je však vo všeobecnosti netriviálny problém.

Okrem generátora $U[0,1]$ budeme používať jednorozmerný generátor z normálneho rozdelenia $N$ s priemerom $0$ a odchýlkou $1$. Vďaka rotačnej symetrickosti normálneho rozdelenia možno generovaním po zložkách pomocou $N$ získať $d$--rozmerné viacrozmerné normálne rozdelenie.
Kvôli centrálnej symetrickosti normálneho rozdelenia možno generovať na $d$--rozmernej sfére a guli vhodným preškálovaním. 
