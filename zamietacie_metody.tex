\section{Zamietacie metódy}

Zamietacie metódy nám poskytujú jeden zo spôsobov rovnomerného generovania bodov na určitej množine.
Myšlienka za nimi je nasledovná: Označme si $X$ množinu, na ktorej chceme rovnomerne náhodne generovať prvky. Predpokladajme, že nevieme priamo rovnomerne generovať body na $X$ ($X$ je veľarozmerná alebo komplikovane zadaná), no vieme rovnomerne generovať na množine $S$, $X \subset S$.

Náš generátor $G_S$ bude pracovať nasledovne:

\begin{algorithm}[H]
	\caption{Zamietacia metóda}
	\label{zamietanie:basic}
	\begin{algorithmic}[1]
		\For {$i=1,\dots,N$}
			\Repeat Vygeneruj bod $\mathbf x^{(i)} \in S$ rovnomerne náhodne
			\Until {$\mathbf x^{(i)} \in X$}
		\EndFor
		\State Vráť ${\mathbf x^{(1)},\mathbf x^{(2)},\dots,\mathbf x^{(N)}}$
	\end{algorithmic}
\end{algorithm}
Generátor $G_S$ vygeneruje bod $\mathbf x \in S$ rovnomerne náhodne, ak je ten bod aj z $X$, tak ho vráti ako výstup, inak vygeneruje nový bod $\mathbf x \in S$. Všimnime si, že generátor $G_S$ je závislý iba od $S$ a že generuje body na $X \cap S=X$ rovnomerne náhodne.

Označme si $\lambda(M)$ objem množiny $M$. Očakávaná rýchlosť generovania závisí od toho, koľkokrát $G_S$ vygeneruje bod mimo $X$. Z rovnomernosti $G_S$ je tá pravdepodobnosť rovná $\frac{\lambda(S-X)}{\lambda(S)} = 1-\frac{\lambda(X)}{\lambda(S)}$. Označme si $p_k$ pravdepodobnosť, že $G_S$ vygeneruje bod z $X$ na $k$--ty pokus, t.j. najprv $k-1$ krát vygeneruje bod mimo $X$ a potom vygeruje bod z $X$. Platí $p_k= (1-\frac{\lambda(X)}{\lambda(S)})^{k-1}\frac{\lambda(X)}{\lambda(S)}$, preto $p_k$ patrí geometrickému rozdeleniu pravdepodobnosti. Očakávý počet generovaní $G_S$ je $E(G_S)=\sum^{\infty}_{0}kp_k=\frac{\lambda(X)}{\lambda(S)} \sum^{\infty}_{0}k(1-\frac{\lambda(X)}{\lambda(S)})^{k-1}=\frac{\lambda(X)}{\lambda(S)} \frac{1}{((1-\frac{\lambda(X)}{\lambda(S)})-1)^2} = \frac{\lambda(S)}{\lambda(X)}$.

Táto metóda generovania je vhodná, ak je $\frac{\lambda(S)}{\lambda(X)}$ dostatočne malé, t.j. ak je obal $S$ relatíve malý oproti polyédru $X$. Ak je $\frac{\lambda(S)}{\lambda(X)} \sim \infty$, tak je táto metóda nepoužiteľná.

Navyše, ak poznáme objem $S$, tak táto metóda nám ako vedľajší produkt poskytne aj štatistické intervaly spoľahlivosti pre objem $X$.

\textbf{TODO v prípade potreby rozpísať}

\subsection{Použitie na generovanie bodu vnútri polyédru}

Zamyslime sa nad tým, ako by sme vedeli použiť túto metódu na generovanie bodu vnútri polyédru. Ako množinu možných $S$, $X \subset S$ môžeme použiť najmenší kváder so stranami rovnobežnými s osami. Vypočítať súradnece kvádra je ľahké, stačí nám to spraviť raz pred (začatím generovania) pomocou lineárneho programovania.

Žiaľ, pre takúto množinu $S$ môže byť podiel $\frac{\lambda(S)}{\lambda(X)}$ byť ľubovoľne veľký. Ako príklad na takú množinu $X$ uveďme kváder s obsahom $k$ pozdĺž diagonály kocky $[0,1]^n$, dotýkajúci sa každej steny kocky $[0,1]^n$. Zrejme najmenšia množina $S$ (kváder so stranami rovnobežnými s osami) obaľujúca $X$ je kocka $[0,1]^n$, ktorá má obsah $1$. Platí $\frac{\lambda(S)}{\lambda(X)}=\frac{1}{k}$. Keďže vieme nájsť kváder taký, že sa dotýka stien kocky $[0,1]^n$ a $k$ je ľubovoľne malé, tak očakávaná dĺžka generovania touto metódou (pre danú množinu $S$) je ľubovovoľne veľká.\\

Ako ďalšia možná množina $S$ prichádza do úvady elipsoid obaľujúci polyéder. Keďže chceme, aby bol podiel $\frac{\lambda(S_{MVEE})}{\lambda(X)}$ čo najmenší, budeme skúmať elipsoid s najmenším obsahom obaľujúci polyéder --- Minumum Volume Enclosing Elipsoid (ďalej MVEE). Môžeme si všimnúť, že MVEE elipsoid obsahuje veľa informácie o tom, ako vyzerá polyéder. Keďže elipsoid je jednotková guľa zobrazená lineárnou transformáciou, možno generovať body vnútri elipsoidu ako obrazy bodov vygenerovaných vnútri jednotkovej gule v lineárnej trasformácii. Nájsť daný elipsoid rýchlo vieme pomocou REX algoritmu, ktorému je venovaná ďalšia kapitola.

Pre túto metódu vieme dokonca odhadnúť rýchlosť generovania. Na odhadnutie veľkosti $\lambda(S_{MVEE})$ možno využiť fakt, že ku každému konvexnému telesu $C$ existuje (unikátny) vpísaný elipsoid s najväčším objemom (takzvaný Johnov elipsoid). Podľa \cite{ellipsoids_ball} Johnov elipsoid pre konvexné $d$--rozmerné teleso $C$ (v našom prípade polyéder $X$) zobrazený rovnoľahlosťou so stredom s centre elipsoidu a koeficientom $d$ obsahuje celé teleso $C$. Tým pádom objem MVEE elipsoid je nanajvýš $d$--krát väčší ako objem Johnovho elipsoidu, preto je nanajvýš $d$--krát väčší ako objem telesa $C$.
$$\frac {\lambda(S_{MVEE})}{\lambda(X)} \leq \frac{d\lambda(S_{JE})}{\lambda(X)} < \frac{d\lambda(X)}{\lambda(X)}=d,$$ kde $S_{JE}$ je Johnov elipsoid prislúchajúci ku polyédru $X$. Očakávaný počet generovaní pomocou nadmnožiny $S_{MVEE}$ je teda najviac počtu rozmerov.\\

Keďže najjednoduchšia množina $S$, v ktorej vieme generovať, je kváder, ako ďalšú uvažovanú množina $S$ možno zvážiť najmenší kváder obaľujúci MVEE elipsoid (bez podmienky, že jeho strany sú rovnobežné s osami sústavy). Zrejme osi kvádra budú zhodné s osami MVEE elipsoidu. Označme si daný kváder $S_K$ a metódu generovania založenú na $S_K$ kvádrova metóda. Pre $S_K$ platí, že je obrazom kocky $[0,1]^n$ v zobrazení, ktoré zobrazí jednotkovú guľu na MVEE elipsoid.

Avšak, keďže tento kváder $S_K$ nie je rovnobežný s osami sústavy, ako prirodzený spôsob rovnomerného generovania vnútri tohoto kvádra možno použiť generovanie vnútri kvádra (resp. kocky) s osami rovnobežnými s osami sústavy a následne zobrazené lineárnou transformáciou.\\

Tento prístup je analogický prístupu s MVEE elipsoidom. Dokonca, lineárne zobrazenia pri $S_K$ a $S_{MVEE}$ sú rovnaké. Rozdiel je jedine v tom, že pri $S_{MVEE}$ rovnomerne generujeme na $N$--rozmernej jednotkovej guli (kde $N$ je dimenzia priestoru), čo následne zobrazujeme lineárnym zobrazením. Pri $S_K$ rovnomerne generujeme na $N$--rozmernej jednotkovej kocke (obaľujúcej $N$--rozmernú jednotkovú guľu), čo zobrazujeme lineárnym zobrazením na kváder $S_K$ obaľujúci MVEE elipsoid.

Ukážme si, že pri dostatočne rýchlom rovnomernom generovaní bodov v guli kvádrova metóda nie je rýchlejšia ako MVEE metóda. Porovnajme očakávaný čas týchto prístupov. Označme si $t_z$ očakávaný čas výpočtu lineárneho zobrazenia, $S_G$ $N$--rozmernú jednotkovú guľu a $t_G$ očakávaný čas vygenerovania bodu v nej, $S_K$ $N$--rozmernú jednotkovú kocku a $t_K$ očakávané čas vygenerovanie bodu v nej. Nakoniec si označme $S_P$ množinu bodov hľadaného polyédra.

Očakávaný čas na vygenerovanie bodu pomocou MVEE elipsoidu je $\frac{\lambda(S_G)}{\lambda(S_P)}t_zt_G=\frac{t_z}{\lambda(S_P)}t_G\lambda(S_G)$, očakávný čas na vygenerovanie pomocou obaľujúceho kvádra $\frac{\lambda(S_K)}{\lambda(S_P)}t_zt_K=\frac{t_z}{\lambda(S_P)}t_K\lambda(S_K)$.
Podiel očakávaných časov je $$\frac{\text{očakávaný čas kvádrovej metódy}}{\text{očakávaný čas MVEE metódy}}=\frac{\frac{\lambda(S_K)}{\lambda(S_P)}t_zt_K}{\frac{\lambda(S_G)}{\lambda(S_P)}t_zt_G}=\frac{\lambda(S_K)t_K}{\lambda(S_G)t_G}.$$ Ak použijeme metódu na rovnomerné generovanie bodov vnútri gule s očakávaným časom $t_G \le \frac{\lambda(S_K)t_K}{\lambda(S_G)}$, tak podiel očakávaných trvaní metód bude menší--rovný ako jedna, teda kvádrová metóda bude pomalšia.

Všimnime si, že zamietacia metóda na generovanie bodov v $S_G$ pomocou nadmnožiny $S_K$ má očakávaný počet generovaní rovný $\frac{\lambda(S_K)t_K}{\lambda(S_G)}$, preto očakávané trvanie vygenerovania bodu pomocou nej je $\frac{\lambda(S_K)t_K}{\lambda(S_G)}$.
Týmto sme ukázali, že ak by sme pri generovali body v polyédri pomocou MVEE metódy, pričom rovnomerné generovanie bodov v $S_G$ by sme realizovali pomocou zamietacej metódy s nadmnožinou $S_K$, dostali by sme presne rýchlosť kvádrovej metódy. Kedže existujú aj rýchlejšie metódy rovnomerného generovania vnútri gule, kvádrová metóda je pomalšia ako MVEE metóda. Ďalej sa ňou nebudeme zaoberať.\\
