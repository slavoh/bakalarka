\chapter{Zamietacie metódy}

Zamietacie metódy nám poskytujú jeden zo spôsobov rovnomerného generovania bodov na určitej množine.
Myšlienka za nimi je nasledovná: Označme si $X$ množinu, na ktorej chceme rovnomerne náhodne generovať prvky. Predpokladajme, že nevieme priamo rovnomerne generovať body na $X$, no vieme rovnomerne generovať na množine $S$, $X \subset S$.

Náš generátor $G_z$ bude pracovať nasledovne:

\begin{algorithm}[H]
	\caption{Zamietacia metóda}
	\label{zamietanie:basic}
	\begin{algorithmic}[1]
		\For {$n=0,\dots,N-1$}
			\Repeat Vygeneruj bod $x^{(n)} \in S$ rovnomerne náhodne
			\Until {$x^{(n)} \in X$}
		\EndFor
		\State Vráť ${x^{(1)},x^{(2)},\dots,x^{(N)}}$
	\end{algorithmic}
\end{algorithm}
Generátor $G_z$ vygeneruje bod $x \in S$, ak je ten bod aj z $X$, tak ho vráti ako výstup, inak vygeneruje nový bod $x \in S$.\\

Generátor $G_z$ generuje na $X$ rovnomerne náhodne. Očakávaná rýchlosť generovania závisí od toho, koľkokrát $G_z$ vygeneruje bod mimo $X$. Z rovnomernosti $G_z$ je tá pravdepodobnosť rovná $\frac{|S-X|}{|S|} = 1-\frac{|X|}{|S|}$. Označme si $p_k$ pravdepodobnosť, že $G_z$ bude generovať bod z $S$ $k$-krát. Platí $p_k= (1-\frac{|X|}{|S|})^{k-1}\frac{|X|}{|S|}$. Očakávý počet generovaní $G_z$ je $E(G_z)=\sum^{\infty}_{0}kp_k=\frac{|X|}{|S|} \sum^{\infty}_{0}k(1-\frac{|X|}{|S|})^{k-1}$.

Táto metóda generovania je vhodná, ak je $\frac{|X|}{|S|}$ dosť veľké, tj. ak je obal $S$ polyédru $X$ dostatočne malý. Ak je $\frac{|X|}{|S|} \sim 0$, tak je táto metóda nepoužiteľná.\\

\textbf{TODO citovat}

\section{Použitie na generovanie bodu vnútri polyédru}

Zamyslime sa nad tým, ako by sme vedeli použiť túto metódu na generovanie bodu vnútri polyédru. Ako množinu možných $S$, $X \subset S$ môžeme použiť najmenší kváder so stranami rovnobežnými s osami. Vypočítať súradnece kvádra je ľahké, stačí nám to spraviť raz pred (začatím generovania) pomocou lineárneho programovanie.

Žiaľ pre takúto množinu $S$ môže byť podiel $\frac{|X|}{|S|}$ byť ľubovoľne malý. Ak bude $X$ kváder s obsahom $k$ pozdĺž diagonály kocky $[0,1]^n$, dotýkajúci sa každej steny kocky $[0,1]^n$, tak najmenšia množina $S$ (kváder so stranami rovnobežnými s osami) je kocka $[0,1]^n$, ktorá má obsah $1$. Platí $\frac{|X|}{|S|}=k$. Keďže vieme nájsť kváder taký, že $k$ je ľubovoľne malé, tak očakávaná dĺžka generovania touto metódou (pre danú množinu $S$) je ľubovovoľne veľká.\\

V rámci tejto práce budeme skúmať použitie problému obalenia polyédru elipsoidom s najmenším obsahom (Minimum Volume Enclosing Elipsoid, ďalej MVEE). Ako množinu $S$ zvolíme nájdený elipsoid. Generovať body v ňom rovnomerne náhodne vieme ľahko pomocou REX algoritmu, s ktorým sa oboznámime neskôr.

Taktiež sa pokúsime pomocou hlavných osí MVEE elipsoidu obaliť polyéder kvádrom s najmenším obsahom (jeho strany nemusia byť rovnobežné s osami sústavi). Ako množinu $S$ môžeme zvoliť daný kváder.

