\section{Zamietacie metódy}

Zamietacie metódy nám poskytujú jeden zo spôsobov rovnomerného generovania bodov na určitej množine.
Myšlienka za nimi je nasledovná: Označme si $X$ množinu, na ktorej chceme rovnomerne náhodne generovať prvky. Predpokladajme, že nevieme priamo rovnomerne generovať body na $X$, no vieme rovnomerne generovať na množine $S$, $X \subset S$.

Náš generátor $G_S$ bude pracovať nasledovne:

\begin{algorithm}[H]
	\caption{Zamietacia metóda}
	\label{zamietanie:basic}
	\begin{algorithmic}[1]
		\For {$i=0,\dots,N-1$}
			\Repeat Vygeneruj bod $x^{(i)} \in S$ rovnomerne náhodne
			\Until {$x^{(i)} \in X$}
		\EndFor
		\State Vráť ${x^{(1)},x^{(2)},\dots,x^{(N)}}$
	\end{algorithmic}
\end{algorithm}
Generátor $G_S$ vygeneruje bod $x \in S$, ak je ten bod aj z $X$, tak ho vráti ako výstup, inak vygeneruje nový bod $x \in S$. Všimnime si, že generátor $G_S$ je závislý iba od $S$, generuje na $X$ rovnomerne náhodne.

Očakávaná rýchlosť generovania závisí od toho, koľkokrát $G_S$ vygeneruje bod mimo $X$. Z rovnomernosti $G_S$ je tá pravdepodobnosť rovná $\frac{|S-X|}{|S|} = 1-\frac{|X|}{|S|}$. Označme si $p_k$ pravdepodobnosť, že $G_S$ vygeneruje bod z $X$ na $k$-ty pokus, t.j. najprv $k-1$ krát vygeneruje bod mimo $X$ a potom vygeruje bod z $X$. Platí $p_k= (1-\frac{|X|}{|S|})^{k-1}\frac{|X|}{|S|}$. Očakávý počet generovaní $G_S$ je $E(G_S)=\sum^{\infty}_{0}kp_k=\frac{|X|}{|S|} \sum^{\infty}_{0}k(1-\frac{|X|}{|S|})^{k-1}=\frac{|X|}{|S|} \frac{1}{((1-\frac{|X|}{|S|})-1)^2} = \frac{|S|}{|X|}$.

Táto metóda generovania je vhodná, ak je $\frac{|S|}{|X|}$ dostatočne malé, t.j. ak je obal $S$ relatíve malý oproti polyédru $X$. Ak je $\frac{|S|}{|X|} \sim \infty$, tak je táto metóda nepoužiteľná.

\subsection{Použitie na generovanie bodu vnútri polyédru}

Zamyslime sa nad tým, ako by sme vedeli použiť túto metódu na generovanie bodu vnútri polyédru. Ako množinu možných $S$, $X \subset S$ môžeme použiť najmenší kváder so stranami rovnobežnými s osami. Vypočítať súradnece kvádra je ľahké, stačí nám to spraviť raz pred (začatím generovania) pomocou lineárneho programovanie.

Žiaľ, pre takúto množinu $S$ môže byť podiel $\frac{|S|}{|X|}$ byť ľubovoľne veľký. Ako príklad na takú množinu $X$ uveďme kváder s obsahom $k$ pozdĺž diagonály kocky $[0,1]^n$, dotýkajúci sa každej steny kocky $[0,1]^n$. Zrejme najmenšia množina $S$ (kváder so stranami rovnobežnými s osami) obaľujúca $X$ je kocka $[0,1]^n$, ktorá má obsah $1$. Platí $\frac{|S|}{|X|}=\frac{1}{k}$. Keďže vieme nájsť kváder taký, že sa dotýka stien kocky $[0,1]^n$ a $k$ je ľubovoľne malé, tak očakávaná dĺžka generovania touto metódou (pre danú množinu $S$) je ľubovovoľne veľká.\\

Ako ďalšia možná množina $S$ prichádza do úvady elipsoid obaľujúci polyéder. Keďže chceme, aby bol podiel $\frac{|S|}{|X|}$ čo najmenší, budeme skúmať elipsoid s najmenším obsahom obaľujúci polyéder - Minumum Volume Enclosing Elipsoid (ďalej MVEE). Nájsť daný elipsoid a generovať body v ňom vieme pomocou REX algoritmu, ktorému je venovaná ďalšia kapitola.

Môžeme si všimnúť, že MVEE elipsoid obsahuje veľa informácie o tom, ako vyzerá polyéder. Keďže nedegenerovaný elipsoid je jednotková guľa zobrazená regulárnou lineárnou transformáciou, vieme pomocou inverznej transformácie zobraziť elipsoid na jednotkovú guľu. Dané zobrazenie možno vypočítať pomocou osí MVEE elipsoidu.\\

Keďže najjednoduchšia množina $S$, v ktorej vieme generovať je kváder, v $3.$ kapitole sa pozrieme na prípad, keď za $S$ zvolíme kváder, ktorého osi budú zhodné s osami MVEE elipsoidu. Na daný kváder sa dá pozerať ako na kváder s najmenším objemom obaľujúcim MVEE elipsoid. Tiež pre neho platí, že je obrazom kocky $[0,1]^n$ v zobrazení, ktoré zobrazí jednotkovú guľu na MVEE elipsoid. Okrem iného, daný kváder je kváder obaľujúci MVEE elipsoid s najmenším objemom. Pre daný kváder možno spraviť odhad veľkosti $\frac{|S|}{|X|}$ (viď príloha).



