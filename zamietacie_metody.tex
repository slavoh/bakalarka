\section{Zamietacie metódy}

Zamietacie metódy nám poskytujú ďalší zo spôsobov rovnomerného generovania bodov na určitej množine.
Myšlienka za nimi je nasledovná: Označme si $S$ množinu, na ktorej chceme rovnomerne náhodne generovať prvky. Predpokladajme, že nevieme priamo rovnomerne generovať body na $S$ ($S$ je veľarozmerná alebo komplikovane zadaná), no vieme rovnomerne generovať na množine $T$, $S \subset T$.

Náš generátor $G_T$ bude pracovať nasledovne:

\begin{algorithm}[H]
	\caption{Zamietacia metóda}
	\label{zamietanie:basic}
	\begin{algorithmic}[1]
		\For {$i=1,\dots,N$}
			\Repeat
			\State Vygeneruj bod $\mathbf x^{(i)} \in T$ rovnomerne náhodne
			\Until {$\mathbf x^{(i)} \in S$}
		\EndFor
		\State Vráť ${\mathbf x^{(1)},\mathbf x^{(2)},\dots,\mathbf x^{(N)}}$
	\end{algorithmic}
\end{algorithm}
Generátor $G_T$ vygeneruje bod $\mathbf x$ na množine $T$ rovnomerne náhodne, ak je ten bod aj z $S$, tak ho vráti ako výstup, inak vygeneruje nový bod $\mathbf x \in T$. Všimnime si, že generátor $G_T$ je závislý iba od $T$ a že generuje body na $S \cap T=S$ rovnomerne náhodne.

Označme si $\lambda(M)$ objem množiny $M$. Očakávaná rýchlosť generovania závisí od toho, koľkokrát $G_T$ vygeneruje bod z $T \setminus S$. Z rovnomernosti $G_T$ je tá pravdepodobnosť rovná $\frac{\lambda(T \setminus S)}{\lambda(T)} = 1-\frac{\lambda(S)}{\lambda(T)}$. Označme si $p_k$ pravdepodobnosť, že $G_T$ vygeneruje bod z $S$ na $k$--ty pokus, t.j. najprv $k-1$ krát vygeneruje bod z $T \setminus S$ a potom vygeruje bod z $S$. Platí $p_k= (1-\frac{\lambda(S)}{\lambda(T)})^{k-1}\frac{\lambda(S)}{\lambda(T)}$, preto $p_k$ zodpovedá geometrickému rozdeleniu pravdepodobnosti. Očakávý počet generovaní $G_T$ je $$E(G_T)=\sum^{\infty}_{k=0}kp_k=\frac{\lambda(S)}{\lambda(T)} \sum^{\infty}_{k=0}k(1-\frac{\lambda(S)}{\lambda(T)})^{k-1}=\frac{\lambda(S)}{\lambda(T)} \frac{1}{((1-\frac{\lambda(S)}{\lambda(T)})-1)^2} = \frac{\lambda(T)}{\lambda(S)}.$$

Táto metóda generovania je vhodná, ak je $\frac{\lambda(T)}{\lambda(S)}$ dostatočne malé, t.j. ak je obal $T$ relatíve malý oproti polyédru $S$. Ak je $\frac{\lambda(T)}{\lambda(S)} \sim \infty$, tak je táto metóda nepoužiteľná.\\

Navyše, ak poznáme objem $T$, tak táto metóda nám ako vedľajší produkt poskytne aj štatistické intervaly spoľahlivosti $I_S$ pre objem $S$. Totiž $\frac{\lambda(S)}{\lambda(T)}$ je presne parameter binomického rozdelenia náhodnej premennej $S$, ktorá zodpovedá vygenerovaní bodu z $S$. Preto možno odhadnúť objem $S$ ako $\lambda(S)=\lambda(T)I_S$.

\subsection{Použitie na generovanie bodu vnútri polyédru}

Zamyslime sa nad tým, ako by sme vedeli použiť túto metódu na generovanie bodu vnútri polyédru. Ako množinu možných $T$, $S \subset T$, môžeme použiť najmenší kváder hranami rovnobežnými s osami sústavy. Vypočítať takýto kváder je ľahké, stačí nám to spraviť raz pred (začatím generovania) pomocou lineárneho programovania.

Žiaľ, pre takúto množinu $T$ môže byť podiel $\frac{\lambda(T)}{\lambda(S)}$ ľubovoľne veľký. Ako príklad na takú množinu $S$ uveďme kváder s obsahom $k$, ktorý ``leží pozdĺž diagonály'' kocky $[0,1]^n$ a dotýka sa každej steny kocky $[0,1]^n$. Zrejme najmenšia množina $T$ (kváder s hranami rovnobežnými s osami sústavy) obaľujúca $S$ je kocka $[0,1]^n$, ktorá má obsah $1$. Platí $\frac{\lambda(T)}{\lambda(S)}=\frac{1}{k}$. Keďže vieme nájsť polyéder (napr. kváder) taký, že sa dotýka stien kocky $[0,1]^n$ a $k$ je ľubovoľne malé, tak očakávaná dĺžka generovania touto metódou (pre danú množinu $T$) môže byť ľubovovoľne veľká.\\

Ako ďalšia možná množina $T$ prichádza do úvahy elipsoid obaľujúci polyéder. Keďže chceme, aby bol podiel $\frac{\lambda(T_{MVEE})}{\lambda(S)}$ čo najmenší, budeme skúmať elipsoid s najmenším obsahom obaľujúci polyéder --- Minumum Volume Enclosing Elipsoid (ďalej MVEE). Môžeme si všimnúť, že MVEE elipsoid obsahuje veľa informácie o tom, ako vyzerá polyéder. Keďže elipsoid je jednotková guľa zobrazená lineárnou transformáciou, možno generovať body vnútri elipsoidu ako obrazy bodov vygenerovaných vnútri jednotkovej gule v lineárnej trasformácii. Nájsť daný elipsoid rýchlo vieme pomocou REX algoritmu, ktorému je venovaná ďalšia kapitola.

Pre túto metódu vieme dokonca odhadnúť rýchlosť generovania. Na odhadnutie veľkosti $\lambda(T_{MVEE})$ možno využiť fakt, že ku každému konvexnému telesu $C$ existuje (unikátny) vpísaný elipsoid s najväčším objemom (takzvaný Johnov elipsoid). Podľa \cite{ellipsoids_ball} Johnov elipsoid pre konvexné $d$--rozmerné teleso $C$ (v našom prípade polyéder $S$) zobrazený rovnoľahlosťou so stredom s centre elipsoidu a koeficientom $d$ obsahuje celé teleso $C$. Zobrazením danou rovnoľahlosťou sa objem $d$--rozmerného telesa zväčší $d^d$--krát. Tým pádom je objem MVEE nanajvýš $d^d$--krát väčší ako objem Johnovho elipsoidu, preto je nanajvýš $d^d$--krát väčší ako objem telesa $C$.
$$\frac {\lambda(T_{MVEE})}{\lambda(S)} \leq \frac{d^d\lambda(T_{JE})}{\lambda(S)} < \frac{d^d\lambda(S)}{\lambda(S)}=d^d,$$ kde $T_{JE}$ je Johnov elipsoid prislúchajúci ku polyédru $S$. Očakávaný počet generovaní pomocou nadmnožiny $T_{MVEE}$ je teda najviac $d^d$ (kde $d$ je počet rozmerov).\\

Poďme sa zamyslieť nad inými množinami $T$. Keďže najjednoduchšia množina $T$, v ktorej vieme generovať, je kváder, ako ďalšiu uvažovanú množinu $T$ možno zvážiť najmenší kváder obaľujúci MVEE elipsoid (bez podmienky, že jeho hrany sú rovnobežné s osami sústavy). Zrejme osi kvádra budú zhodné s osami MVEE elipsoidu. Označme si daný kváder $T_K$ a metódu generovania založenú na $T_K$ \textit{kvádrova metóda}. Pre $T_K$ platí, že je obrazom kocky $[0,1]^n$ v zobrazení, ktoré zobrazí jednotkovú guľu na MVEE elipsoid.

Avšak, keďže tento kváder $T_K$ nie je rovnobežný s osami sústavy, ako prirodzený spôsob rovnomerného generovania vnútri tohoto kvádra možno použiť generovanie vnútri jednotkovej hyperkocky s osami rovnobežnými s osami sústavy a následne zobrazené lineárnou transformáciou.\\

Tento prístup je analogický prístupu s MVEE elipsoidom. Dokonca, lineárne zobrazenia pri $T_K$ a $T_{MVEE}$ sú rovnaké. Rozdiel je jedine v tom, že pri $T_{MVEE}$ rovnomerne generujeme na $d$--rozmernej jednotkovej guli (kde $d$ je dimenzia priestoru), čo následne zobrazujeme lineárnym zobrazením. Pri $T_K$ rovnomerne generujeme na $d$--rozmernej jednotkovej kocke (obaľujúcej $d$--rozmernú jednotkovú guľu), čo zobrazujeme lineárnym zobrazením na kváder $T_K$ obaľujúci MVEE elipsoid.

Ukážme si, že pri dostatočne rýchlom rovnomernom generovaní bodov v guli kvádrova metóda nie je rýchlejšia ako MVEE metóda. Porovnajme očakávaný čas týchto prístupov. Označme si $t_f$ očakávaný čas výpočtu lineárneho zobrazenia, $T_G$ $d$--rozmernú jednotkovú guľu a $t_G$ očakávaný čas vygenerovania bodu v nej, $T_K$ $d$--rozmernú jednotkovú kocku a $t_K$ očakávané čas vygenerovanie bodu v nej. Nakoniec si označme $T_P$ množinu bodov hľadaného polyédra.

Očakávaný čas na vygenerovanie bodu pomocou zamietacej metódy pomocou MVEE je $\frac{\lambda(T_G)}{\lambda(T_P)}t_ft_G=\frac{t_f}{\lambda(T_P)}t_G\lambda(T_G)$, očakávný čas na vygenerovanie pomocou kvádrovej metódy je $\frac{\lambda(T_K)}{\lambda(T_P)}t_ft_K=\frac{t_f}{\lambda(T_P)}t_K\lambda(T_K)$.
Podiel očakávaných časov je $$\frac{\text{očakávaný čas kvádrovej metódy}}{\text{očakávaný čas MVEE metódy}}=\frac{\frac{\lambda(T_K)}{\lambda(T_P)}t_ft_K}{\frac{\lambda(T_G)}{\lambda(T_P)}t_ft_G}=\frac{\lambda(T_K)t_K}{\lambda(T_G)t_G}.$$ Ak pri MVEE metóde použijeme na rovnomerné generovanie bodov vnútri gule s očakávaným časom $t_G \le \frac{\lambda(T_K)t_K}{\lambda(T_G)}$, tak podiel očakávaných trvaní metód bude menší--rovný ako jedna, teda kvádrová metóda bude pomalšia.

Všimnime si, že zamietacia metóda na generovanie bodov v $T_G$ pomocou nadmnožiny $T_K$ má očakávaný počet generovaní rovný $\frac{\lambda(T_K)t_K}{\lambda(T_G)}$, preto očakávané trvanie vygenerovania bodu pomocou nej je $\frac{\lambda(T_K)t_K}{\lambda(T_G)}$.
Týmto sme ukázali, že ak by sme generovali body v polyédri pomocou MVEE metódy, pričom rovnomerné generovanie bodov v $T_G$ by sme realizovali pomocou zamietacej metódy s nadmnožinou $T_K$, dostali by sme presne rýchlosť kvádrovej metódy. Kedže existujú aj rýchlejšie metódy rovnomerného generovania vnútri gule, kvádrová metóda je pomalšia ako MVEE metóda. Ďalej sa ňou nebudeme zaoberať.

Treba podotknúť, že zamietacia metóda na generovanie bodov v $T_G$ pomocou $T_K$ v $d$ rozmeroch má zásadný problém, že pomer objemov $\frac{\lambda(T_K)}{\lambda(T_G)}$ je už pre malé $d$ obrovský. Totiž v $d$--rozmeroch je $\lambda(T_K)=2^d$ a $\lambda(T_G)=\Gamma(\frac d 2 + 1)$ (kde $\Gamma$ je Eulerova Gamma funkcia), preto 
$$\frac{\lambda(T_K)}{\lambda(T_G)}=\frac{2^d}{\Gamma(\frac d 2 + 1)} \sim \frac{2^d}{\frac{d+1}{2}!}\sim \frac{2^{d}}{\sqrt{(d+1)\pi}(\frac{d+1}{2e})^{(d+1)/2}}=\frac{2^d (2e)^{(d+1)/2}}{\sqrt{(d+1)\pi}(d+1)^{(d+1)/2}}.$$
V čítateli je $c^d$ (pre vhodnú konštantu $c$), menovateľ je rádovo $d^{d/2}$, čo rastie so zväčšujúcou sa dimenziou priestoru asymptoticky oveľa rýchlejšie ako čítateľ.
