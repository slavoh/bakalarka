\section{Zamietacie metódy}

Zamietacie metódy nám poskytujú jeden zo spôsobov rovnomerného generovania bodov na určitej množine.
Myšlienka za nimi je nasledovná: Označme si $X$ množinu, na ktorej chceme rovnomerne náhodne generovať prvky. Predpokladajme, že nevieme priamo rovnomerne generovať body na $X$ ($X$ je veľarozmerná alebo komplikovane zadaná), no vieme rovnomerne generovať na množine $S$, $X \subset S$.

Náš generátor $G_S$ bude pracovať nasledovne:

\begin{algorithm}[H]
	\caption{Zamietacia metóda}
	\label{zamietanie:basic}
	\begin{algorithmic}[1]
		\For {$i=1,\dots,N$}
			\Repeat Vygeneruj bod $x^{(i)} \in S$ rovnomerne náhodne
			\Until {$x^{(i)} \in X$}
		\EndFor
		\State Vráť ${x^{(1)},x^{(2)},\dots,x^{(N)}}$
	\end{algorithmic}
\end{algorithm}
Generátor $G_S$ vygeneruje bod $x \in S$ rovnomerne náhodne, ak je ten bod aj z $X$, tak ho vráti ako výstup, inak vygeneruje nový bod $x \in S$. Všimnime si, že generátor $G_S$ je závislý iba od $S$ a že generuje body na $X \cap S=X$ rovnomerne náhodne.

Očakávaná rýchlosť generovania závisí od toho, koľkokrát $G_S$ vygeneruje bod mimo $X$. Z rovnomernosti $G_S$ je tá pravdepodobnosť rovná $\frac{|S-X|}{|S|} = 1-\frac{|X|}{|S|}$. Označme si $p_k$ pravdepodobnosť, že $G_S$ vygeneruje bod z $X$ na $k$--ty pokus, t.j. najprv $k-1$ krát vygeneruje bod mimo $X$ a potom vygeruje bod z $X$. Platí $p_k= (1-\frac{|X|}{|S|})^{k-1}\frac{|X|}{|S|}$. Očakávý počet generovaní $G_S$ je $E(G_S)=\sum^{\infty}_{0}kp_k=\frac{|X|}{|S|} \sum^{\infty}_{0}k(1-\frac{|X|}{|S|})^{k-1}=\frac{|X|}{|S|} \frac{1}{((1-\frac{|X|}{|S|})-1)^2} = \frac{|S|}{|X|}$.

Táto metóda generovania je vhodná, ak je $\frac{|S|}{|X|}$ dostatočne malé, t.j. ak je obal $S$ relatíve malý oproti polyédru $X$. Ak je $\frac{|S|}{|X|} \sim \infty$, tak je táto metóda nepoužiteľná.

Navyše, ak poznáme objem $S$, tak táto metóda nám ako vedľajší produkt poskytne aj štatistické intervaly spoľahlivosti pre objem $X$.

\textbf{TODO rozpísať}

\subsection{Použitie na generovanie bodu vnútri polyédru}

Zamyslime sa nad tým, ako by sme vedeli použiť túto metódu na generovanie bodu vnútri polyédru. Ako množinu možných $S$, $X \subset S$ môžeme použiť najmenší kváder so stranami rovnobežnými s osami. Vypočítať súradnece kvádra je ľahké, stačí nám to spraviť raz pred (začatím generovania) pomocou lineárneho programovania.

Žiaľ, pre takúto množinu $S$ môže byť podiel $\frac{|S|}{|X|}$ byť ľubovoľne veľký. Ako príklad na takú množinu $X$ uveďme kváder s obsahom $k$ pozdĺž diagonály kocky $[0,1]^n$, dotýkajúci sa každej steny kocky $[0,1]^n$. Zrejme najmenšia množina $S$ (kváder so stranami rovnobežnými s osami) obaľujúca $X$ je kocka $[0,1]^n$, ktorá má obsah $1$. Platí $\frac{|S|}{|X|}=\frac{1}{k}$. Keďže vieme nájsť kváder taký, že sa dotýka stien kocky $[0,1]^n$ a $k$ je ľubovoľne malé, tak očakávaná dĺžka generovania touto metódou (pre danú množinu $S$) je ľubovovoľne veľká.\\

Keďže najjednoduchšia množina $S$, v ktorej vieme generovať, je kváder, uvažujme prípad, keď $S$ je najmenší kváder obaľujúci MVEE elipsoid (bez podmienky, že jeho strany sú rovnobežné s osami sústavy). Zrejme osi kvádra budú zhodné s osami MVEE elipsoidu. Označme si daný kváder $S_K$ a metódu generovania založenú na $S_K$ kvádrova metóda. Pre $S_K$ platí, že je obrazom kocky $[0,1]^n$ v zobrazení, ktoré zobrazí jednotkovú guľu na MVEE elipsoid.

Avšak, keďže tento kváder $S_K$ nie je rovnobežný s osami sústavy, ako prirodzený spôsob rovnomerného generovania vnútri tohoto kvádra možno použiť generovanie vnútri kvádra (resp. kocky) s osami rovnobežnými s osami sústavy a následne zobrazené lineárnou transformáciou.\\

Tento prístup je analogický prístupu s MVEE elipsoidom. Dokonca, lineárne zobrazenia pri $S_K$ a $S_{MVEE}$ sú rovnaké. Rozdiel je jedine v tom, že pri $S_{MVEE}$ rovnomerne generujeme na $N$--rozmernej jednotkovej guli (kde $N$ je dimenzia priestoru), čo následne zobrazujeme lineárnym zobrazením. Pri $S_K$ rovnomerne generujeme na $N$--rozmernej jednotkovej kocke (obaľujúcej $N$--rozmernú jednotkovú guľu), čo zobrazujeme lineárnym zobrazením na kváder $S_K$ obaľujúci MVEE elipsoid.

Ukážme si, že pri dostatočne rýchlom rovnomernom generovaní bodov v guli kvádrova metóda nie je rýchlejšia ako MVEE metóda. Porovnajme očakávaný čas týchto prístupov. Označme si $t_z$ očakávaný čas výpočtu lineárneho zobrazenia, $S_G$ $N$--rozmernú jednotkovú guľu a $t_G$ očakávaný čas vygenerovania bodu v nej, $S_K$ $N$--rozmernú jednotkovú kocku a $t_K$ očakávané čas vygenerovanie bodu v nej. Nakoniec si označme $S_P$ množinu bodov hľadaného polyédra.

Očakávaný čas na vygenerovanie bodu pomocou MVEE elipsoidu je $\frac{|S_G|}{|S_P|}t_zt_G=\frac{t_z}{|S_P|}t_G|S_G|$, očakávný čas na vygenerovanie pomocou obaľujúceho kvádra $\frac{|S_K|}{|S_P|}t_zt_K=\frac{t_z}{|S_P|}t_K|S_K|$.
Podiel očakávaných časov je $$\frac{\text{očakávaný čas kvádrovej metódy}}{\text{očakávaný čas MVEE metódy}}=\frac{\frac{|S_K|}{|S_P|}t_zt_K}{\frac{|S_G|}{|S_P|}t_zt_G}=\frac{|S_K|t_K}{|S_G|t_G}.$$ Ak použijeme metódu na rovnomerné generovanie bodov vnútri gule s očakávaným časom $t_G \le \frac{|S_K|t_K}{|S_G|}$, tak podiel očakávaných trvaní metód bude menší--rovný ako jedna, teda kvádrová metóda bude pomalšia.

Všimnime si, že zamietacia metóda na generovanie bodov v $S_G$ pomocou nadmnožiny $S_K$ má očakávaný počet generovaní rovný $\frac{|S_K|t_K}{|S_G|}$, preto očakávané trvanie vygenerovania bodu je $\frac{|S_K|t_K}{|S_G|}$.
Týmto sme ukázali, že ak by sme pri generovali body v polyédri pomocou MVEE metódy, pričom rovnomerné generovanie bodov v $S_G$ by sme realizovali pomocou zamietaciu metódu s nadmnožinou $S_K$, dostali by sme presne rýchlosť kvádrovej metódy. Kedže existujú aj rýchlejšie metódy rovnomerného generovania vnútri gule, kvádrová metóda je pomalšia ako MVEE metóda. Ďalej sa ňou nebudeme zaoberať.\\



Ako ďalšia možná množina $S$ prichádza do úvady elipsoid obaľujúci polyéder (označme si ho $S_{MVEE}$). Keďže chceme, aby bol podiel $\frac{|S_{MVEE}|}{|X|}$ čo najmenší, budeme skúmať elipsoid s najmenším obsahom obaľujúci polyéder --- Minumum Volume Enclosing Elipsoid (ďalej MVEE). Môžeme si všimnúť, že MVEE elipsoid obsahuje veľa informácie o tom, ako vyzerá polyéder. Keďže elipsoid je jednotková guľa zobrazená lineárnou transformáciou, možno generovať body vnútri elipsoidu ako obrazy bodov vygenerovaných vnútri jednotkovej gule v lineárnej trasformácii. Nájsť daný elipsoid vieme pomocou REX algoritmu, ktorému je venovaná ďalšia kapitola.\\

