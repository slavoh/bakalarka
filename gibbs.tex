\section{Gibbsov generátor}

V tejto sekcii sa budeme zaoberať Gibbsovým generátorom, metódou generovania z triedy MCMC vhodnou na generovanie vo viacrozmernom priestore. Na Gibbsov generátor sa možno dívať ako na špeciálny prípad Metropolis-Hastings algoritmu.

Našou úlohou je generovať z $K$-rozmernej distribúcie $Q$, pričom z $Q$ nevieme generovať priamo. Predpokladajme, že nevieme použiť priamo Metropolis-Hastings algoritmus, lebo $Q(x^{(i)})=Q(x^{(i)}_1, x^{(i)}_2, \dots, x^{(i)}_K)$ je príliš zložitá na generovanie. Taktiež predpokladajme, že ak $Q(x^{(i)})$ obmedzíme na jeden rozmer, tak v ňom vieme generovať rýchlo, tj. možno generovať rýchlo z $Q(x^{(i)}_j | x^{(i+1)}_1, x^{(i+1)}_2 \dots, x^{(i+1)}_{j-1}, x^{(i)}_{j+1}, x^{(i)}_{j+2} \dots, x^{(i)}_K)$.

Gibbsov generátor bude fungovať nasledovne:

\begin{algorithm}[H]
	\caption{Gibbsov generátor \cite{mcmc_intro_mackay}}
	\label{gibbs}
	\begin{algorithmic}[1]
		\State inicializu $x^{(0)} = (x^{(0)}_1, x^{(0)}_2, \dots, x^{(0)}_K)$
		\For {$i=1,\dots,N$}
			\For {$j=0,1,\dots,K$}
				\State $x^{(i)}_j \sim Q(x_j | x^{(i+1)}_1, x^{(i+1)}_2 \dots, x^{(i+1)}_{j-1}, x^{(i)}_{j+1}, x^{(i)}_{j+2} \dots, x^{(i)}_K)$
			\EndFor
			\State $x^{(i+1)}= (x^{(i+1)}_1, x^{(i+1)}_2 \dots, x^{(i+1)}_K)$
		\EndFor
		\State Vráť ${x^{(1)},x^{(2)},\dots,x^{(N)}}$
	\end{algorithmic}
\end{algorithm}

Gibbsov generátor ako špeciálny prípad Metropolis-Hastings algoritmu má podobné vlastnosti ako Metropolis-Hastings algoritmus.\\

\textbf{TODO} specifickost oproti všeobecnému Metropolis-Hastingsu: menej parametrov

\textbf{TODO} prakticke vyuzitie

