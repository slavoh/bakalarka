\subsection{Gibbsov generátor}

V tejto podsekcii sa budeme zaoberať Gibbsovým generátorom, metódou generovania z triedy MCMC vhodnou na generovanie vo viacrozmernom priestore.
Našou úlohou je generovať z $n$--rozmernej distribúcie $Q$, pričom z $Q$ nevieme generovať priamo. Predpokladajme, že nevieme použiť Hit--and--Run generátor, lebo $Q(x^{(i)})=Q(x^{(i)}_1, x^{(i)}_2, \dots, x^{(i)}_n)$ je kvôli veľkému rozmeru priestoru príliš zložitá na generovanie. Pred podrobným popísaním vlastností algoritmu si najprv ukážme, ako Gibbsov generátor funguje.

Gibbsov generátor určí kandidátsku hustotu $Q(x^{(i)})$ tak, že bude možné generovať z $Q(x^{(i)})$ po súradniciach.
Gibbsov generátor bude generovať bod $(x^{(i+1)})=(x^{(i+1)}_1, x^{(i+1)}_2, \dots, x^{(i+1)}_n)$ postupne po súradniciach, $j$--tú súradnicu $x^{(i+1)}_j$ vygeneruje ako $$Q(x^{(i+1)}_j | x^{(i+1)}_1, x^{(i+1)}_2 \dots, x^{(i+1)}_{j-1}, x^{(i)}_{j+1}, x^{(i)}_{j+2} \dots, x^{(i)}_n).$$

Označme si cieľový polyéder $S$ a $d_{i,j}$ priamku rovnobežnú s $j$--tou osou prechádzajúcou cez bod $(x^{(i+1)}_1, \dots, x^{(i+1)}_{j}, x^{(i)}_{j+1}, \dots, x^{(i)}_n)$.
V prípade rovnomerného generovania na polyédroch je hustota $Q(x^{(i+1)}_j | x^{(i+1)}_1, \dots, x^{(i+1)}_{j-1}, x^{(i)}_{j+1}, \dots, x^{(i)}_n)$ je rovnomerná na úsečke $d_{i,j} \cap S$ a nulová inde.

Gibbsov generátor funguje nasledovne:

\begin{algorithm}[H]
	\caption{Gibbsov generátor \cite{mcmc_intro_mackay}}
	\label{gibbs}
	\begin{algorithmic}[1]
		\State inicializu $x^{(0)} = (x^{(0)}_1, x^{(0)}_2, \dots, x^{(0)}_n)$
		\For {$i=0,\dots,N-1$}
			\For {$j=0,1,\dots,n$}
				\State $x^{(i)}_j \sim Q(x^{(i+1)}_j | x^{(i+1)}_1, x^{(i+1)}_2 \dots, x^{(i+1)}_{j-1}, x^{(i)}_{j+1}, x^{(i)}_{j+2} \dots, x^{(i)}_n)$
			\EndFor
			\State $x^{(i+1)}= (x^{(i+1)}_1, x^{(i+1)}_2 \dots, x^{(i+1)}_n)$
		\EndFor
		\State Vráť ${x^{(1)},x^{(2)},\dots,x^{(N)}}$.
	\end{algorithmic}
\end{algorithm}

Gibbsov generátor predpokladá, že možno rýchlo generovať jednotlivé súradnice z rozdelenia $Q(x^{(i+1)}_j | x^{(i+1)}_1, \dots, x^{(i+1)}_{j-1}, x^{(i)}_{j+1}, \dots, x^{(i)}_n)$.

V prípade rovnomerného generovania v polyédri je daný predpoklad splnený, vďaka linearite nerovníc pri H--reprezentácii polyédra možno ľahko rovnomerne generovať na úsečke $d_{i,j} \cap S$. Hraničné body úsečky vieme vypočítať veľmi rýchlo pomocou lineárneho programovania, čiže generovať z týchto rozdelení je obzvlášť jednoduché.

Ako špeciálny prípad Metropolis--Hastings algoritmu má Gibbsov generátor podobné vlastnosti ako Metropolis--Hastings algoritmus. Jeho hlavnou výhodou je, že je jednoduchý a neobsahuje žiadne parametre.\\

Môžeme si všimnúť, že pri zväčšovaní počtu rozmerov priestoru rastie čas potrebný na vygenerovanie čas potrebný na výpočet programu rastie asymptoticky kvadraticky.

\textbf{TODO v prípade potreby rozpísať viac, inak zmazať}