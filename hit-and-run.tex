\subsection{Hit--and--Run generátor}

Ako jedna z možností na realizáciu Metropolis--Hastings algoritmu prichádza do úvahy Hit--and--Run generátor. Algoritmus je analogický s algoritmom Metropolis--Hasting.

Označme si $S$ zadaný polyéder. Kandidátska hustota $q(\cdot|\mathbf x^{(i)})$ je určená priamkou $\mathbf d_i$ s náhodným smerom cez bod $\mathbf x^{(i)}$. Hustota $q(\cdot|\mathbf x^{(i)})$ je konštantná na úsečke $S \cap \mathbf d_i$ a nulová inde. Pri danej hustote je markovovský reťazec časovo reverzibilný, t.j. $q(\mathbf x^{(i+1)}|\mathbf x^{(i)})=q(\mathbf x^{(i)}|\mathbf x^{(i+1)}),$ \cite{hit-and-run_chen}, preto je funkcia $\alpha$ konštantne $1$. Algoritmus sa teda každým krokom pohne do ďalšieho bodu.

Hit--and--Run generátor funguje nasledovne:

\begin{algorithm}[H]
	\caption{Hit--and--Run generátor \cite{hit-and-run_chen},\cite{zhluky_lukacek}}
	\label{hit--and--run}
	\begin{algorithmic}[1]
		\State Inicializuj $\mathbf x^{(0)}$
		\For {$i=0,\dots,N-1$}
			\State Vygeneruj smer $\mathbf d_i$ z distribúcie $D$ na povrchu sféry
			\State Nájdi množinu $S_i(\mathbf d_i,\mathbf x^{(i)})=\{\lambda \in \mathbb{R}; \mathbf x^{(i)} + \lambda \mathbf d_i \in S \} $
			\State Vygeneruj $\lambda_i \in S_i$ podľa hustoty $q(\lambda | \mathbf d_i, \mathbf x^{(i)})$
			\State Nastav $\mathbf x^{(i+1)}=\mathbf x^{(i)}+\lambda_i \mathbf d_i$
		\EndFor
		\State Vráť $\mathbf x^{(1)},\mathbf x^{(2)},\dots,\mathbf x^{(N)}$.
	\end{algorithmic}
\end{algorithm}

Použiteľnosť Hit--and--Run generátora závisí od toho, ako rýchlo vieme generovať smery $\mathbf d_i$ z rozdelenia $D$. Ak by dimenzia priestoru bola príliš veľká, nebolo by možné generovať rýchlo z rozdelenia $D$, a preto celý algoritmus by bol pomalý. 

Na rovnomerné vygenerovanie $d$--rozmerného smerového vektoru vygenerujeme bod v $d$--rozmernom rotačne symetrickom rozdelení a následne ho zobrazíme z počiatku sústavy na jednotkovú sféru. Takto dostaneme rovnomerné rozdelenie na $d$--rozmernej sfére. Takto zložitosť vygenerovania smeru rastie len asymptoticky lineárne s dimenziou, preto aj rýchlosť generovania bodov rastie len lineárne s dimenziou.

Podľa \cite{spheres_harman} ako $d$--rozmerné rotačné symetrické rozdelenie možno zvoliť mnohorozmerné normálne rozdelenie so zložkami z nezávislých jednorozmerných normálnych rozdelení so strednou hodnotou $0$ a rozptylom $1$.
