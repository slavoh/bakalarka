\chapter{Metódy generovania vnútri polyédru}

V tejto kapitole sa budeme zaoberať známymi metódami na generovanie z určitého rozdelenia, ktoré je v našom prípade konštantné vnútri polyédra a nulové mimo polyédra. V prvej podkapitole sa budeme zaoberať triedou Metropolis--Hastings algoritmov, v druhej podkapitole sa budeme zaoberať zamietacími metódami.

\section{Metropolis--Hastings metódy}

V tejto podkapitole si predstavíme triedu Metropolis--Hastings algoritmov na generovanie bodov z ľubovoľného rozdelenia. Na postupnosť bodov generovnaných algoritmami z triedy Metropolis--Hastings sa dá pozerať ako na postupnosť stavov markovovského reťazca (angl. ``Markov Chain Monte'' Carlo, ďalej MCMC). Pri danej cieľovej hustote $Q$ je Markovovský reťazec konštruovaný tak, aby mal jediné stacionárne rozdelenie, ktoré je totožné s $Q$. Pri použití vhodného nastavenia parametrov algoritmu bude rozdelenie vygenerovaných bodov s rastúcim počtom bodov konvergovať ku žiadanému stacionárnemu rozdeleniu $Q$.

V nasledujúcej časti stručne predstavíme všeobecný (abstraktný) Metropolis--Hastings algoritmus, podrobnejšie vysvetlenie možno nájsť napríklad v \cite{metropolis-hastings_chib}. V následných častiach predstavíme Hit--and--Run generátor a Gibbsov generátor ako jeho konkrétne realizácie.

\subsection{Všeobecný Metropolis--Hastings algoritmus}

Majme cieľovú hustotu $Q$, z ktorej chceme generovať; v prípade rovnomerného generovania vnútri polyédra je konštantná v polyédri a nulová mimo neho.

Metropolis--Hastings algoritmus \cite{metropolis-hastings_chib} sa nachádza v stave $\mathbf x^{(i)}$ reprezentovanom bodom v polyédri, pričom stav určuje \textit{kandidátsku hustotu} $q(\cdot | \mathbf x^{(i)})$ závislú na $\mathbf x^{(i)}$. Táto kandidátska hustota (angl. ``proposal density'') je volená tak, aby z nej bolo možné jednoducho generovať ďalšie body. Môže byť značne odlišná od cieľovej hustoty $Q$, avšak je nutné, aby limitné rozdelenie vygenerovaných bodov konvergovalo ku $Q$.

Algoritmus postupuje iteratívne, v jednom kroku vygeneruje ďalší potenciálny stav $\mathbf y$ podľa hustoty $q(\cdot |\mathbf x^{(i)})$. Ďalší stav algoritmu $\mathbf x^{(i+1)}$ bude $\mathbf y$ s pravdepodobnosťou $\alpha (\mathbf y|\mathbf x^{(i)})$ (algoritmus sa ``pohne"), inak to bude $\mathbf x^{(i)}$ (algoritmus ``ostane stáť''). Funkcia $\alpha$ sa označuje ako \textit{pomer akceptovania} (angl. ``acceptance ratio''), je definované ako $$\alpha (\mathbf y|\mathbf x^{(i)}) = \text{min}\left( 1, \frac{Q(\mathbf y)}{Q(\mathbf x)} \frac {q(\mathbf x|\mathbf y)}{q(\mathbf y | \mathbf x)} \right).$$ 
Pravdepodobnosť $\alpha (\mathbf y|\mathbf x^{(i)})$ môže byť vo všeobecnosti zložitá. V prípade, že je hustota $Q$ rovnomerná a postupnosť stavov markovovského reťazca časovo reverzibilná, t.j. $$q(\mathbf x^{(i+1)}|\mathbf x^{(i)})=q(\mathbf x^{(i)}|\mathbf x^{(i+1)}),$$ tak je daná pravdepodobnosť pohybu konštante jedna, $\alpha (\mathbf y|\mathbf x^{(i)})=1$. Tieto predpoklady budú splnené pri ďalej spomínaných algoritmoch (Hit--and--Run generátore a Gibbsovom generátore). Teda naše MCMC metódy, ktoré budeme používať pre generovanie z rovnomerného rozdelenia, sa v každom kroku ``pohnú", na rozdiel od všeobecných MCMC generátorov, ktoré často ``stoja", lebo neakceptujú kandidátov $\mathbf y$ na prechod.

\begin{algorithm}[H]
	\caption{Všeobecný Metropolis--Hastings algoritmus \cite{metropolis-hastings_chib}}
	\label{metropolis-hastings}
	\begin{algorithmic}[1]
		\State inicializuj $\mathbf x^{(0)}$
		\For {$i=0,1,\dots,N-1$} 
			\State Vygeneruj bod $\mathbf y$ z $q(\cdot|\mathbf x^{(i)})$
			\State Vygeneruj $u$ z $U(0,1)$.
			\If {$u \leq \alpha(\mathbf y | \mathbf x^{(i)})$}
				\State Nastav $\mathbf x^{(i+1)}=\mathbf y$
			\Else
				\State Nastav $\mathbf x^{(i+1)}=\mathbf x^{(i)}$
			\EndIf
		\EndFor
		\State Vráť ${\mathbf x^{(1)},\mathbf x^{(2)},\dots,\mathbf x^{(N)}}$.
	\end{algorithmic}
\end{algorithm}

Môžeme si všimnúť, že v Metropolis--Hastings algoritmoch, je hustota bodu $\mathbf x^{(i)}$ závislá od predchádzajúceho bodu $\mathbf x^{(i-1)}$. Podľa \cite{metropolis-hastings_chib} je pri vhodnej voľbe kandidátskej hustoty $q(\cdot|\mathbf x^{(i)})$ a pravdepodobnosti $\alpha$ možné dokázať, že napriek závislosti po sebe idúcich bodov je pre $N \rightarrow \infty$ limitné rozdelenie náhodneho vektora $\mathbf x^{(N)}$ rovné $Q$. Potrebná veľkosť $N$ na dosiahnutie dostatočne presného odhadu hustoty $Q$ sa nazýva burn--in period.

V ďalších častiach si ukážeme niekoľko konkrétnych realizácii Metropolis--Hastings algoritmu. Každá z tých metód obsahuje určité predpoklady na distribúciu, z ktorej chceme generovať, no dá použiť aj na rovnomerné generovanie bodov v polyédri.