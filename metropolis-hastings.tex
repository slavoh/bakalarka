\chapter{Metódy generovania vnútri polyédru}

V tejto kapitole sa budeme zaoberať známymi metódami na generovanie z určitého rozdelenia, ktorá je v našom prípade rovnomerná vnútri polyédra a nulová mimo polyédra. V prvej podkapitole sa budeme zaoberať triedou Metropolis--Hastings algoritmov, v druhej podkapitole sa budeme zaoberať zamietacími metódami.

\section{Metropolis--Hastings metódy}

V tejto sekcii si predstavíme triedu Metropolis--Hastings algoritmov na generovanie bodov z ľubovoľného rozdelenia. Na postupnosť bodov generovnaných algoritmami z triedy Metropolis--Hastings sa dá pozerať ako na postupnosť stavov markovovských reťazcov (Markov Chain Monte Carlo, ďalej MCMC). Daným spôsobom je ich možné analyzovať a dokázať, že generujú body podľa žiadaného rozdelenia (kvôli časovému obmedzeniu práce je táto časť vynechaná).

V nasledujúcej časti predstavíme  všeobecný Metropolis--Hastings algoritmus, podrobnejšie vysvetlenie viete nájsť napríklad v \cite{metropolis-hastings_chib}. V následných častiach opíšeme jeho konkrétne realizácie.

\subsection{Všeobecný Metropolis--Hastings algoritmus}

Majme cieľovú hustotu $Q$ z ktorej chceme generovať, v prípade rovnomerného generovania vnútri polyédra je rovnomerná v polyédri a nulová mimo neho.

Metropolis--Hastings algoritmus \cite{metropolis-hastings_chib} sa nachádza v stave $x^{(i)}$ reprezentovanom bodom v polyédri, stav určuje \textit{kandidátsku hustotu} $Q(x^{(i)})$ závislú na $x^{(i)}$. Táto kandidátska hustota (proposal density) je volená tak, aby sa z nej bolo možné jednoducho generovať a môže byť značne odlišná od cieľovej hustoty $Q$.

Algoritmus postupuje iteratívne, v jednom kroku vygeneruje ďalší potenciálny stav $y$ podľa hustoty $Q(x^{(i)})$. Ďalší stav algoritmu $x^{(i+1)}$ bude $y$ s pravdepodobnosťou $\alpha (y|x^{(i)})$, inak to bude $x^{(i)}$.

\begin{algorithm}[H]
	\caption{Všeobecný Metropolis--Hastings algoritmus \cite{metropolis-hastings_chib}}
	\label{metropolis-hastings}
	\begin{algorithmic}[1]
		\State inicializuj $x^{(0)}$
		\For {$i=0,1,\dots,N$} 
			\State Vygeneruj bod $y$ z $Q(x^{(i)})$
			\State Vygeneruj $u$ z $U(0,1)$.
			\If {$u \leq \alpha(y | x^{(i)})$}
				\State Nastav $x^{(i+1)}=y$
			\Else
				\State Nastav $x^{(i+1)}=x^{(i)}$
			\EndIf
		\EndFor
		\State Vráť ${x^{(1)},x^{(2)},\dots,x^{(N)}}$.
	\end{algorithmic}
\end{algorithm}

Môžeme si všimnúť, že v Metropolis--Hastings algoritme je bod $x^{(i)}$ závislý od predchádzajúceho bodu $x^{(i-1)}$. Podľa \cite{metropolis-hastings_chib} je pri vhodnej voľbe kandidátskej hustoty $Q(x^{(i)})$ a pravdepodobnosti $\alpha$ možné dokázať, že napriek závislosti po sebe idúcich bodov je pre $N \rightarrow \infty$ limitné rozdelenie náhodneho vektora $x^{(N)}$ rovné $Q$. Potrebná veľkosť $N$ na dosiahnutie dostatočne presného odhadu hustoty $Q$ sa nazýva burn--in period.

V ďalších častiach si ukážeme niekoľko konkrétnych realizácii Metropolis--Hastings algoritmu. Každá z tých metód obsahuje určité predpoklady na distribúciu, z ktorej chceme generovať, no dá použiť aj na rovnomerné generovanie bodov v polyédri.