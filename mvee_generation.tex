\section {Rovnomerné generovanie bodov v polyédri pomocou MVEE elipsoidu}

V tejto podkapitole sa budeme zaoberať, ako pomocou už vypočítaného MVEE elipsoidu možno rovnomerne generovať body vnútri zadaného polyédra. Budeme používať zamietaciu metódu, vygerujeme bod v elipsoide a overíme, či je daný bod tiež v polyédre.

Nakoľko každý elipsoid je jednotková guľa zobrazená lineárnym zobrazením, na rovnomerné generovanie v MVEE elipsoide najprv rovnomerne vygenerujeme bod v jednotkovej guli a následne ho zobrazíme daným lineárnym zobrazením do bodu v MVEE elipsoide. Rovnomernosť generovania sa zachová \textbf{TODO prečo}.

Na rovnomerné generovanie bodu $x$ v $d$--rozmernej jednotkovej guli so stredom v nule najprv vygenerujeme $d$--rozmerný uhol rovnomerne náhodne a následne vygenerujeme vzdialenosť bodu $x$ od nuly tak, aby mali oblasti rôzne vzdialené od $0$ rovnakú pravdepodobnosť na vygenerovanie.

Na rovnomerné vygenerovanie $d$--rozmerného uhlu vygenerujeme bod v $d$--rozmernom centrálne symetrickom rozdelení (napríklad normálnom) a následne ho zobrazíme na jednotkovú sféru. Takto zrejme dostaneme rovnomerné rozdelenie na $d$--rozmernej sfére. Následne vygenerujeme vzdialenosť od nuly podľa rozdelenia daného funkciou $f:[0,1] \rightarrow [0,1]$: $$f(x)=\frac{x^d}{\int_0^1 y^d dy}$$.
Pri danej funkcii má generovaný bod $x$ rovnakú pravdepodobnosť, že padne do ľubovoľne vzdialenej oblasti s rovnakým obsahom \textbf{TODO prečo}, preto je dané rozdelenie vnútri gule rovnomerné.