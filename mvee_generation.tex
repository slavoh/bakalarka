\section {Rovnomerné generovanie bodov v polyédri pomocou MVEE elipsoidu}

V tejto podkapitole sa budeme bližšie zaoberať MVEE metódou, ako pomocou už vypočítaného MVEE elipsoidu možno rovnomerne generovať body vnútri zadaného polyédra. Budeme používať zamietaciu metódu, vygerujeme bod v elipsoide a overíme, či je daný bod tiež v polyédre.

Nakoľko každý elipsoid je jednotková guľa zobrazená lineárnym zobrazením, na rovnomerné generovanie v MVEE elipsoide najprv rovnomerne vygenerujeme bod v jednotkovej guli a následne ho zobrazíme daným lineárnym zobrazením do bodu v MVEE elipsoide. Kedže zobrazenie je lineárne, jeho jakobián je konštantný, preto rovnomernosť hustoty generovania nezávisí od $x$. A teda rovnomernosť generovania sa zachová.\\
\label{generovanie_v_mvee}
Na rovnomerné generovanie bodu $\mathbf x$ v $d$--rozmernej jednotkovej guli so stredom v nule najprv vygenerujeme $d$--rozmerný smerový vektor rovnomerne náhodne a následne vygenerujeme vzdialenosť bodu $\mathbf x$ od nuly tak, aby mali oblasti rôzne vzdialené od $0$ rovnakú pravdepodobnosť na vygenerovanie.

Na rovnomerné vygenerovanie $d$--rozmerného smerového vektoru vygenerujeme bod v $d$--rozmernom centrálne symetrické rozdelenie a následne ho zobrazíme na jednotkovú sféru (rovnako ako pri generovaní smeru Hit--and--Run generátora). Následne vygenerujeme vzdialenosť od nuly podľa rozdelenia daného hustotou $h:[0,1] \rightarrow [0,1]$: $$h(\mathbf x)=\frac{\mathbf x^d}{\int_0^1 y^d dy}$$
Pri generovaní z danej hustoty získame rovnomerné rozdelenie na guli. Zobrazenie bodu $\mathbf x$ z $d$--rozmernej sféry do $d$--rozmernej gule možno vyjadriť pomocou generátora $U(0,1)$ z rovnomerného rozdelenia na $(0,1)$ ako $$f(\mathbf x)=\mathbf x\cdot U(0,1)^{\frac{1}{d}}$$.

Pri danej funkcii má generovaný bod $\mathbf x$ rovnakú pravdepodobnosť, že padne do ľubovoľne vzdialenej oblasti s rovnakým obsahom, dané rozdelenie je rovnomerné.